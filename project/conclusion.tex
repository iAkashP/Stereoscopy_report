\chapter{Conclusion and Future Scope}
\section{Conclusion}
\paragraph{}This distance
measurement is based upon the pictures taken from two
horizontally displaced cameras. The user should select the
object on left camera and the algorithm finds similar object
on the right camera. From displacement of the same object
on both pictures, the distance to the object can be calculated.
Although the method is based on relatively simple
algorithm, the calculated distance is still quite accurate.
Better results are obtained with wider base (distance between
the cameras). This is all according to theoretical derivations.
\begin{itemize}
	\item Visible light is used as mode of communication, hence no hustle of Radio Frequencies.
	\item This is an accurate way of calculating depth of object in passive manner.
	\item It has wide range of application, from medical science to research.
	\item Now a days, it is implemented in Smartphone cameras to improvise autofocus.
	
\end{itemize}
\newpage
\section{Future Scope}
\paragraph{}Recent advances in 3-dimensional (3-D) stereoscopic imaging have enabled 3-D display technologies in the operating room. We find 2 beneficial applications for the inclusion of 3-D imaging in clinical practice. The first is the real-time 3-D display in the surgical theater, which is useful for the neurosurgeon and observers. In surgery, a 3-D display can include a cutting-edge mixed-mode graphic overlay for image-guided surgery. The second application is to improve the training of residents and observers in neurosurgical techniques. This article documents the requirements of both applications for a 3-D system in the operating room and for clinical neurosurgical training, followed by a discussion of the strengths and weaknesses of the current and emerging 3-D display technologies. An important comparison between a new autostereoscopic display without glasses and current stereo display with glasses improves our understanding of the best applications for 3-D in neurosurgery. Today's multiview autostereoscopic display has 3 major benefits: It does not require glasses for viewing; it allows multiple views; and it improves the workflow for image-guided surgery registration and overlay tasks because of its depth-rendering format and tools. Two current limitations of the autostereoscopic display are that resolution is reduced and depth can be perceived as too shallow in some cases. Higher-resolution displays will be available soon, and the algorithms for depth inference from stereo can be improved. The stereoscopic and autostereoscopic systems from microscope cameras to displays were compared by the use of recorded and live content from surgery. To the best of our knowledge, this is the first report of application of autostereoscopy in neurosurgery.