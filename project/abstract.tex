\begin{center}
\thispagestyle{empty}
\vspace{2cm}
\LARGE{\textbf{ABSTRACT}}\\[1.0cm]
\end{center}
\thispagestyle{empty}
\large{\paragraph{}A stereoscopic motion or still picture in which the right component of a composite image usually red in color is superposed on the left component in a contrasting color to produce a \emph{three-dimensional effect} when viewed through correspondingly colored filters in the form of spectacles.}
\large{\paragraph{}
	The modes of 3D presentation you are most familiar with are the paper glasses with red and blue lenses. The technology behind 3D, or stereoscopic, movies is actually pretty simple. They simply recreate the way humans see normally.}
\large{\paragraph{}
	Since your eyes are about two inches apart, they see the same picture from slightly different angles. Your brain then \emph{correlates these two images} in order to gauge distance. This is called \emph{binocular vision}, this process by presenting each eye with a slightly different image.}
\large{\paragraph{}The binocular vision system relies on the fact that our two eyes are spaced about 2 inches (5 centimeters) apart. Therefore, each eye sees the world from a slightly different perspective, and the binocular vision system in your brain uses the difference to calculate distance. Your brain has the ability to correlate the images it sees in its two eyes even though they are slightly different.}
\large{\paragraph{}The cross-eyed viewing method swaps the left and right eye images so that they will be correctly seen cross-eyed, the left eye viewing the image on the right and vice versa. The fused three-dimensional image appears to be smaller and closer than the actual images, so that large objects and scenes appear miniaturized. \emph{This method is usually easier for freeviewing novices}. As an aid to fusion, a fingertip can be placed just below the division between the two images, then slowly brought straight toward the viewer's eyes, keeping the eyes directed at the fingertip; at a certain distance, a fused three-dimensional image should seem to be hovering just above the finger.}\\\\
\textbf{Keywords: } Three-dimensional effect, binocular vision, depth.