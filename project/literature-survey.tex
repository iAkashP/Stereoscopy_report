\chapter{History}
\large{\paragraph{}In \emph{280 A.D., Euclid} was the first to recognize that depth perception is obtained when each eye simultaneously receives one of two dissimilar images of the same object. 
	In 1584 \textbf{Leonado da Vinci} studied the perception of depth and, unlike most of contemporaries, produced paintings and sketches that showed a clear understanding of shading, texture and viewpoint projection.Around the year 1600, \textbf{Giovanni Battista della Porta} produced the first artificial 3-D drawing based on Euclid’s notions on how 3-D perception by humans works.}
\large{\paragraph{}
\textbf{Queen Victoria} visited the World's Fair in London in 1851 and was so entranced by the stereoscopes on display that she precipitated an enthusiasm for three-dimensional photography that soon made it a popular form of entertainment world-wide.}
\large{\paragraph{}It was \textbf{Sir Charles Wheatstone} who in 1833 first came up with the idea of presenting slightly different images to the two eyes using a device he called a reflecting mirror stereoscope. When viewed stereoscopically, he showed that the two images are combined in the brain to produce 3-D depth perception. The invention of the Brewster Stereoscope by the Scottish scientist \textbf{Sir David Brewster} in 1849 provided a template for all later stereoscopes. This in turn stimulated the mass production of stereo photography which flourished alongside \emph{mono-photography}. Stereo photography peaked around the turn of the century and went out of fashion as movies increased in popularity.}